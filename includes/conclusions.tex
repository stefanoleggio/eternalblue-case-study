\section{Conclusions}
Due to his potential, EternalBlue was implemented in a lot of ransomware like WannaCry. Between 2017
and 2018 its spread all over the word was incredible. Just in 2017 were infected more than 200,000 computers in 150 countries.
Also in Italy there was many cases of this ransomware, in particular the entire network of Milano Bicocca University was compromised.\\
\begin{figure}[ht!]
    \centering
      \includegraphics[scale=0.3]{images/bicocca.png}
      \caption{Milano Bicocca infected by WannaCry}
\end{figure}

\noindent The National Health Service hospitals in England and Scotland was one of the largest entity compromised by this ransomware. It was particullary 
easy spread the infection in their network because many hospitals shared files with SMB each other over internet.
\begin{figure}[ht!]
    \centering
      \includegraphics[scale=0.3]{images/bbc.png}
      \caption{BBC article about WannaCry}
\end{figure}

\begin{figure}[ht!]
    \centering
      \includegraphics[scale=0.3]{images/theverge.png}
      \caption{The Verge article about WannaCry}
\end{figure}

\noindent Cyence, a cyber risk modeling firm, has estimated the total losses due to WannaCry at \$4 billion, making it one
of the most damaging cyber attack.
