\section{Introduction}
\subsection{History}
EternalBlue is an exploit developed by the National Security Agency of United States targeting Windows systems\cite{nsa-hacked}.
In 2017 it was leaked by the Shadow Brockers hacker group\cite{bbc-eternalblue} along with other 35 exploits and hacking
tools, the most relevant are:
\begin{itemize}
    \item Fuzzbunch: An exploitation framework like Metasploit
    \item DanderSpritz: Command and control solution for the post exploitation
    \item DoublePulsar: Trojan
    \item EternalBlue: Service Message Block (SMB) protocol exploit
\end{itemize}

\subsection{Relevance}
EternalBlue exploits a vulnerability that has affected almost all the Windows versions, from XP to 10th version, providing full remote code execution\cite{microsoft-bulletin}.
Nowadays it is considered as one of the biggest leak ever happened to a national security agency.\\
To make the things worse, after the leak on 14th April 2017, EternalBlue was used in Ransomware\cite{exploit-wannacryptor} and Crypto Miner due to the large number of vulnerable devices.
The most famouses cases are:
\begin{itemize}
    \item WannaCry: Ransomware
    \item Adylkuzz: Cryptominer
\end{itemize}
The spread of these viruses was incredible, they were able to infect a network with a thousand of devices in a few minutes.