\section{Memory analysis}
In thi section it will be explained how the SMB buffers are stored in the Windows systems. This is very important in order to know 
what the attacker is overflowing and how he can be able to execute the code.
\subsection{Srvnet buffers}
A srvnet buffer is a struct used to store the content of smb transactions, it is composed by three components:
\begin{itemize}
    \item srvnet\_wsk\_struct: It contains a table of functions pointers, they are executed by the server machine in some specific occasion, for example
    when the connection is stopped it executes the function SrvNetWskReceiveComplete() to process the data in the buffer.
    \item Memory Description List (MDL): A kernel structure that maps the data buffer to the physical memory fragments.
    \item The buffer data itself
\end{itemize}

\begin{figure}[ht!]
    \centering
      \includegraphics[]{images/srvnet_buffer.png}
      \caption{Srvnet buffer}
\end{figure}

\subsection{Hal's Heap}
The Windows Hardware Abstraction Layer (HAL) is a module that runs in kernel mode and it is loaded on Windows boot. It consists in a set of routines to access hardware resources
through programming interface. This is used to allow programmers to write software which is hardware-indipendent.\\
This is a particular heap in Windows, its addres has been kept always costant. The Address Space Layout Randomization (ASLR), which consists in the randomization of the memory addresses,
doesn't regards the ones in the HAL heap which are always costant.
Fortunately, in the last versions of Windows 10 also the HAL's heap is randomized, but not at the time of the exploit.
Another important feature of the HAL's heap is that it has execution permission is some Windows version.\\
It is clear now that this part of memory is perfect for the exploitation, here is where the attacker wants to execute the payload.

\begin{figure}[ht!]
    \centering
      \includegraphics[scale=0.5]{images/hal.png}
      \caption{Hardware description layer}
\end{figure}

\subsection{Grooming technique}